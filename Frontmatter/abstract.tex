%----------------------------------------------------------------------------------------
%	ABSTRACT PAGE
%----------------------------------------------------------------------------------------
\begin{abstract}
\addchaptertocentry{\abstractname} % Add the abstract to the table of contents
Since their discovery over 50 years ago, a vast amount of work has been done to understand the complex radio emission mechanisms of pulsars. While their properties are stable on long timescales spanning many thousands of rotations, an abundance of fascinating, dynamic phenomena have been discovered by studying emission on short timescales of single pulses, with polarisation information being key to understanding the physical processes involved. This thesis presents a series of studies of the radio emission of four different pulsars from different segments of the population, with particular focus on their single-pulse modulation and polarisation properties. 

PSR~B0031$-$07 exhibits three states with drifting subpulses of different periodicities, and is one of several pulsars known to show orthogonal polarisation mode (OPM) transitions that are modulated together with the total intensity emission. To explain this, I propose a model based on pulse longitude-dependent magnetospheric mixing and attenuation of the two linear OPMs. By exploring an atlas of possible geometries, the asymmetry in polarisation and total intensity is qualitatively reproduced by an axisymmetric carousel of sub-beams, while asymmetries occur due to propagation through the magnetosphere. This demonstrated that the carousel model is still a viable explanation for drifting subpulses.

Secondly, the properties of PSR~J1926$-$0652 are studied. This pulsar was discovered with the Five-hundred-metre Aperture Spherical radio Telescope (FAST) during commissioning, and was observed with the Parkes telescope as well. The single pulses of PSR~J1926$-$0652 detected with FAST reveal somewhat unstable drifting subpulses and a moderate nulling fraction. The twin-peaked profile of this pulsar is shown to be significantly different to the average profile in the last active pulses immediately prior to a null. This suggests that drifting subpulses and nulling are associated processes, which would not be a natural consequence of existing models.

Thirdly, the millisecond pulsar PSR~J1518+4904 is studied with FAST, which is sensitive enough to clearly detect its single pulses and also to discover two new profile components at 1250~MHz. Fourier analysis of the single-pulse data shows that multiple periodicities are present in different profile components, including one associated with the main profile that has a pulse longitude-dependent fluctuation frequency. Such behaviour has never been observed in any pulsar before. No evidence is found for mode changing, leading to the conclusion that these single pulse periodicities occur simultaneously. 

Finally the multi-frequency observations of the slow pulsar PSR~J0250+5854 are presented, which were performed simultaneously with FAST, two LOFAR international stations, and NenuFAR. The spectral coverage of this object is extended by a factor of ${\sim}5$, revealing a steep radio spectrum with a low-frequency turnover. The profile is shown to broaden at higher frequencies, counter to the expectation of radius-to-frequency mapping. Polarisation data help to confirm that this is because only a small fraction of the open field lines produce radio emission at the lowest frequencies.

\end{abstract}