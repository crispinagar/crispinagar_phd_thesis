\chapter[A broadband radio study of PSR~J0250+5854]{A broadband radio study of PSR~J0250+5854: the slowest-spinning radio pulsar known}
\label{chapt: J0250}

In this chapter I present simultaneous observations of the most slowly rotating known radio pulsar, PSR~J0250+5854, (period $P_1 = 23.5$~s) with FAST, two LOFAR international stations (UK608 at Chilbolton and DE601 at Effelsberg), and NenuFAR. The detections of this pulsar at 1250~MHz (FAST) and 57~MHz (NenuFAR) are the highest and lowest frequency published detections respectively to date, increasing the spectral coverage of this object by a factor of five. A flux density of $4\pm2$~$\upmu$Jy was measured at 1250~MHz giving an exceptionally steep spectral index of $-3.5^{+0.2}_{-1.5}$. The spectrum also has a turnover below $\sim95$~MHz. In conjunction with previous observations of this pulsar with the GBT and LOFAR Core, I show that the intrinsic profile width increases towards higher frequencies, contrary to the predictions of conventional radius-to-frequency mapping. I use polarimetric data from FAST and LOFAR Core to constrain the emission geometry of PSR~J0250+5854, leading to the conclusion that its polar cap emission is produced at an absolute height of several hundreds of kilometres, similar to the other rotation-powered non-recycled pulsars, supporting the argument that emission height is relatively constant across the population. Finally, I draw a comparison between PSR~J0250+5854 and the other slow rotation powered pulsars,  and contrast these with the radio-detected magnetars. I conclude that although they have similar spin periods, the magnetars have intrinsically wider radio beams that the slow rotation-powered pulsars. Consequently, the lower beaming fraction of the latter is what makes objects such as PSR~J0250+5854 so scarce.

\section{Introduction}
\label{sec: J0250 - introduction}

Pulsars are all rapidly rotating, highly magnetised neutron stars; however, some pulsars rotate significantly less rapidly than others. In this chapter I discuss observations of PSR~J0250+5854, a radio pulsar with an extraordinarily long period of $P_1 = 23.5$~s that was discovered by \citet{TBC+2018} in the Low Frequency Array (LOFAR) Tied-Array All-Sky Survey \citep[LOTAAS;][]{SCB+2019}. It was initially detected with the LOFAR High Band Antenna array between 110 and 190~MHz, and in observations with the Green Bank Telescope (GBT) between 300 and 400~MHz. It is the longest-period radio pulsar discovered to data, more than twice the period of the second slowest-spinning \citep[PSR~J2251$-$3711 at $P_1 = 12.1$~s;][]{MKE+2020}, and almost three times slower than the well-studied 8.5~s pulsar PSR~J2144$-$3933 \citep{YMJx1999}. Finding such a slow pulsar is rare, which may be explained by the fact that slow pulsars have a lower beaming fraction \citep[the fraction of the sky covered by the radiation beam; e.g.][]{ECxx1989} which makes it less likely that they are visible to an Earthbound observer. Additionally, there are practical issues with detecting slow pulsars due to selection effects making them significantly harder to identify in pulsar survey data if only a small number of pulses are present. The presence of red noise in periodicity searches further hinders their identification \citep[e.g.][]{LBH+2015,HKRx2017}.

Furthermore, older, slower pulsars lose their ability to create electron-positron pairs and accelerate them sufficiently to produce the detectable coherent radio emission \citep{Sxxx1971}. This fact means that the continued emission of a pulsar as slow as PSR~J0250+5854 is curious: as discussed in \citet{TBC+2018} it lies in a relatively empty part of the $P$-$\dot{P}$-diagram (see Fig.~\ref{fig: intro - ppdot diagram} in Sec.~\ref{sec: intro - general intro - pulsar population}), beyond the `death valley' as defined by \citet{CRxx1993} and the vacuum-gap curvature radiation death line proposed by \citet{ZHMx2000}. The 8.5~s pulsar PSR~J2144$-$3933 also lies beyond the death line -- although it rotates almost three times faster than PSR~J0250+5854, its much smaller spin-down rate makes it more constraining for death line models \citep{MBMA2020}. Nevertheless, the fact that these pulsars are still able to produce detectable radio emission is argued to be compatible with the partially screened gap model \citep[e.g.][]{Sxxx2013}.

The extreme period of PSR~J0250+5854 places it in the company of the magnetars, which have periods of around 2 to 12~s \citep{OKxx2014}, but its spin-down rate is $\dot{P}=2.72\times 10^{-14}$~s~s$^{-1}$ is around a thousand times smaller than those objects. It lies close to the parameter space inhabited by the population of X-ray Dim Isolated Neutron Stars (XDINSs). Of the seven brightest XDINSs, five have strong, dipolar magnetic fields which may mean they are related to the magnetars in their emission properties \citep[][see Sec~\ref{sec: intro - general intro - pulsar population}]{Hxxx2007, KKxx2007}. These objects are detected only as soft thermal X-ray sources, without radio counterparts. However, to date PSR~J0250+5854 remains undetected in X-rays, despite a dedicated \textit{Swift} X-ray Telescope observation, which makes it difficult to confirm a connection between it and XDINSs (see \citealt{TBC+2018} for details). Similarly, it has not shown any magnetar-like behaviour such as bursts, or large radio variability, as of yet.

In this work I present the first detection of PSR~J0250+5854 at radio frequencies between 1 and 1.5~GHz using the Five-hundred-metre Aperture Spherical Radio Telescope (FAST), along with the UK608 and DE601 LOFAR international stations, and NenuFAR (New Extension in Nan\c{c}ay Upgrading loFAR). The NenuFAR detection at 57~MHz is the lowest frequency detection of PSR~J0250+5854 published to data, and the FAST detection the highest, resulting in an extension by a factor of $\sim55$ in spectral coverage of this unique source in the radio domain. The radio frequency evolution is a key part of understanding the pulsar emission mechanism, and is particularly relevant for pulsars such as PSR~J0250+5854 which are on the cusp of the death line. Multi-wavelength observations can provide information on features including hte spectral index (how the flux of the pulsar changes with frequency), and changes in the shape and polarisation properties of the radio beam.

Measurements of radio spectra from a large population of pulsars began in earnest with \citet{Sxxx1973}, \citet{MMxx1980}, and \citet{IKMS1981}, using frequencies around and below 100~MHz. Most pulsars were found to have steep spectra which could be modelled with a simple power law $S_\nu \propto \nu^k$, where $S_\nu$ is the mean flux density at some frequency $\nu$ and $k$ is the spectral index. Some pulsars were found to deviate from this relation, including the identification of a turn-over at low frequencies which can be attributed to absorption mechanisms, while others show a cut-off at high frequencies due to a steepening, or break, in the spectrum \citep{Sxxx1973}. Two recent population studies of radio pulsar spectral indices have been performed by \citet{BKK+2016} and \citet{JSK+2018}. \citet{BKK+2016} studied the spectra of 165 non-recycled pulsars in the Northern sky observed with LOFAR and found that 124 were best fitted by the simple power law model, whilst the remaining 41 were fitted with a broken power law. They found a mean spectral index of $k = -1.4$. Similarly, \citet{JSK+2018} studied 441 pulsars using the Parkes radio telescope over centre frequencies of 728, 1382, and 3100~MHz. They found that 79~per~cent obeyed a simple power law relation. The spread of spectral indices in their sample is described by a shifted log-normal distribution with a weighted mean of $-1.60\pm0.03$ and a standard deviation of 0.54. \citet{TBC+2018} were able to detect PSR~J0250+5854 over the frequency range between 120 and 350~MHz, and fitted a spectral index of $-2.6\pm0.5$, which is on the steeper side of the population distribution. However, with no detection of the pulsar at higher frequencies (1484~MHz, 1532~MHz) using the Lovell and Nan\c{c}ay radio telescopes (respectively), nor a detection with the core LOFAR Low Band Antenna stations (55~MHz), uncertainty remained over the broadband shape of the radio spectrum. 

Frequency evolution of the integrated pulse profile can also be used to infer the shape and structure of the emission regions. With its period of 23.5~s, PSR~J0250+5854 has an extremely large light-cylinder (with a radius $R_\mathrm{LC} = cP/2\pi =  1.123\times10^{6}$~km, where $c$ is the speed of light), and hence a tiny polar cap which connects to the open field line region (see Figs.~\ref{fig: intro - basic geometry} and \ref{fig: intro - emission cone} in Sec.~\ref{sec: intro - general intro}). The diameter of the polar cap is $D_\mathrm{PC} \approx 2R\sqrt{R/R_\mathrm{LC}} \approx~60$~m, where $R = 10$~km is the canonical neutron star radius \citep[e.g.][]{Sxxx1971}. By comparison, a typical pulsar with a period $P = 0.5$~s would have a much larger polar cap of $D_\mathrm{PC} \approx 410$~m. This implies that for typical emission heights of hundreds of kilometres, the radio beam of PSR~J0250+5854, and hence the duty cycle of its radio pulse, can be expected to be very narrow. Indeed \citet{TBC+2018} reported a pulse width of only $\sim$1$\degr$ at 129, 168, and 350~MHz. 

The shapes of pulse profiles after correcting for propagation effects are in general observed to be frequency-dependent. Often, the profile width decreases with increasing frequency which suggests that higher-frequency emission is produced lower in the magnetosphere. This correlation is known as radius-to-frequency mapping (RFM hereafter). RFM was first theorised by \citet{RSxx1975} who related the emission height to the local plasma frequency in the magnetosphere. The electron density, hence plasma frequency, is expected to decrease with increasing altitude ($\rho \propto r^{-3}$), thereby predicting that the radio beam expands with decreasing frequency. A number of pulsars have been found to deviate from this relation \citep[e.g.][]{Txxx1991, CWxx2014, PHS+2016} -- this suggests that not necessarily the same magnetic field lines are active at all frequencies (or emission heights), resulting in the appearance and disappearance of profile components with observing frequency \citep[e.g.][]{Cxxx1978, MRxx2002}. This can obfuscate the geometrical interpretation of measured profile widths, and is discussed further in Sec.~\ref{sec: J0250 - discussion - width}. Radio polarisation data can help in disentangling these effects, which we further explore for PSR~J0250+5854 in Sec.~\ref{sec: J0250 - analysis - polarisation and geometry}, although degeneracies often remain \citep[e.g.][]{KJW+2010}.

% The structure of this paper is as follows. In Section \ref{sec: J0250 - observations} the new observations are described, followed by a brief explanation of the radio-frequency interference (RFI) excision techniques used. The analysis of the data described in Section \ref{sec: J0250 - analysis} is divided into three parts: pulse profile evolution with frequency, polarisation properties, and the spectral shape of the pulsar flux density. These results are then discussed in a broader context in Section \ref{sec: J0250 - discussion}, and our conclusions are summarised in Section~\ref{sec: J0250 - conclusions}.


% \section{Observations}
% \label{sec: J0250 - observations}

% As part of a shared-risk proposal, PSR~J0250+5854 was observed on the 22nd May 2019 with FAST. FAST is a Chinese mega-science facility built and operated by the National Astronomical Observatories, Chinese Academy of Sciences \citep{NLJ+2011, LWQ+2018}. With an effective aperture of 300~m in diameter, it is the world's largest single-dish radio telescope, and is located in a natural depression in Guizhou Province. The central beam of the high-performance 19-beam receiver operating between 1 and 1.5~GHz \citep{JTH+2020} was used. Two LOFAR international stations -- UK608 (United Kingdom) and DE601 (Germany) --- and NenuFAR (France) provided overlapping observations. Chilbolton is home to the UK LOFAR station UK608, formally known as the Rawlings Array and the DE601 station is located at Effelsberg\footnote{DE601 was operated in the stand-alone GLOW (German Long Wavelength consortium) mode at the time of our observations.}. They each consist of two sub-arrays: the High Band Antenna (HBA; 110--240~MHz) and Low Band Antenna (LBA; 10--90~MHz) \citep{HWG+2013, SHA+2011}, although only the HBA were used in this project and the bandwidth was limited to 110--190~MHz. NenuFAR at the time of our observations consisted of 52 groups of 19 dual-polarised antennas, operating between 10 and 85~MHz \citep{ZDT+2020}. It is located alongside and extends the capabilities of the Nan\c{c}ay LOFAR station (FR606). Table \ref{tab: observations} gives a summary of the overlapping observations conducted. The LOFAR international stations were observing over the full duration of the FAST observations, and in the case of NenuFAR significantly longer.

% \begin{table}
% 	\centering
% 	\caption[Simultaneous observations of PSR~J0250+5854]{Observation properties of the simultaneous observations. There is a full overlap of the data for the period during which FAST was recording data for PSR~J0250+5854 on the 22nd May 2019. `No. Longitude bins' is the number of samples per pulse period.}
% 	\label{tab: observations}
% 	\begin{tabular}{crrrrrrr} % 6 columns, alignment for each
% 		\hline
% 	    Observation & Centre freq. & Bandwidth & No. freq. channels & Start time & No. pulses & Length & No. longitude bins\\
% 	    & (MHz) & (MHz) & & (UTC) & & (hh:mm:ss) & \\
% 		\hline
% 		FAST        & $1250.00$ & $500.0$   & 4096 & 02:34:07  & 100   & 00:39:13  & 8192 \\
% 		FAST	    & $1250.00$ & $500.0$   & 4096 & 03:31:23 & 153   & 01:00:01  & 8192 \\
% 		DE601        & $158.55$  & $71.4$    & 488  & 01:56:19  & 417   & 02:43:34  & 1024   \\
% 		UK608  & $149.71$  & $95.2$    & 1952 & 02:02:50  & 382   & 02:29:50  & 8192  \\
% 		NenuFAR     & $56.54$   & $75.0$    & 384  & 02:03:18  & 1528  & 09:59:21  & 2048   \\
% 		\hline
% 	\end{tabular}
% \end{table}

% Prior to observing PSR~J0250+5854, FAST performed a $\sim$15-minute observation of a well-known, bright pulsar PSR~J0139+5814 to validate the set-up of the observing system. A noise diode signal was injected into the FAST multibeam receiver to facilitate polarisation calibration. PSR~J0250+5854 was observed for two consecutive hours, interspersed with noise diode observations. Finally, the BL Lacertae object J0303+472 \citep{VVxx2006} was observed for purposes of flux calibration -- this source was chosen due to its proximity to PSR~J0250+5854.

% All data were folded and de-dispersed with \textsc{dspsr} \citep{SBxx2011} using the ephemeris and dispersion measure (DM) reported in \citet{TBC+2018} to form a pulse sequence. De-dispersion was done coherently for the DE601, UK608, and NenuFAR data, and incoherently for the FAST data. Flux and polarisation calibration were done using the \textsc{pac} program in \textsc{psrchive} \citep{HSMx2004}. Further processing made use of \textsc{psrsalsa} \citep{Wxxx2016}\footnote{\url{https://github.com/weltevrede/psrsalsa}}.

% Although not part of the simultaneous observations, this project also made use of an observation of PSR~J0250+5854 using the LOFAR Core stations \citep{HWG+2013} conducted on 28 October 2017. This was part of a run of observations conducted by \citet{TBC+2018}, but is a different data set to the profiles shown in that paper. This LOFAR Core observation was similarly processed with \textsc{psrchive}.

% \subsection{Data cleaning techniques}
% \label{sec: J0250 - observations - cleaning}

% PSR~J0250+5854 has a low flux-density which, in combination with its extraordinarily long period, makes the analysis susceptible to radio-frequency interference (RFI) that affects the baseline level during a rotation period. A somewhat different approach to mitigate the effects of RFI was taken for the different datasets, guided by the nature of the RFI.

% In all datasets the worst-affected frequency channels were identified and excluded from further analysis. The FAST data were affected by stochastic baseline variations that persisted throughout the observation, with a scale somewhat larger than the pulsar's duty cycle. These baseline variations were removed by subtracting sinusoids plus a constant offset fitted to the off-pulse region for each rotation of the star in each frequency and Stokes parameter independently. This ensures that the mean intensity of the off-pulse region is zero.  The sinusoids were harmonics of the pulse period, and were fitted up to the 23\textsuperscript{rd} harmonic. These sinusoids have periods which significantly exceed the duty cycle of the pulsar, hence the shapes of the pulses were not affected by this process.

% The RFI in the DE601 and UK608 observations was very different, appearing as short, bright, impulsive spikes orders of magnitude brighter than the pulsar signal. An effective approach to mitigation was to iteratively clip the brightest samples for each rotation of the star and each frequency channel individually. The clipping is done conservatively to ensure that the pulsar signal is unaffected. With the worst RFI suppressed, the remaining RFI and baseline variations were reduced using the same method described for the FAST data.

% The NenuFAR data were recorded during the commissioning phase of the instrument with a coherent de-dispersion pipeline \citep[LUPPI;][]{BGT+2020} operating in single-pulse mode. The observations were folded with \textsc{dspsr} and a polynomial of degree two was subtracted from the baseline of each sub-integration to suppress the effect of bandpass variations. The data from two frequency bands were appended after correcting for the appropriate delay\footnote{The observation was recorded in two sub-bands which have a slight offset when dedispersed, and so had be aligned manually. This is only necessary for older observations conducted before the ``early science'' phase.}. The observation was cleaned using a modified version of \textsc{coastguard} \citep{LKG+2016} and finally dealt with in the same way as with the FAST data.

% \section{Analysis and results}
% \label{sec: J0250 - analysis}

% \subsection{Profile morphology and width evolution}
% \label{sec: J0250 - analysis - profile widths}

% Figure \ref{fig: profiles} shows the integrated pulse profile of PSR~J0250+5854 as observed by the four telescopes in order of descending frequency. We also include the profile observed by the GBT at 350~MHz from \citet{TBC+2018}, and a LOFAR Core detection at 149~MHz (an observation from 28 October 2017). The profiles for the LOFAR international stations and NenuFAR are obtained from the full-length observation rather than only the overlap period with the FAST observation to increase the signal-to-noise ratio. This is motivated by the fact that there is no evidence that the profile shapes were changing during these observations. 

% \begin{figure}
%     \includegraphics[width=\columnwidth]{Figures/J0250/profiles.png}
%     \caption[Multifrequency profiles of PSR~J0250+5854]{The pulse profiles of PSR~J0250+5854 at different radio frequencies. The top profile is from FAST (1250~MHz), followed by GBT (350~MHz), then DE601 (154~MHz), UK608 (150~MHz), LOFAR Core (149~MHz), and finally NenuFAR (57~MHz). The FAST, DE601, and UK608 profiles are overlapping in time, and are aligned using the known DM and after accounting for geometric delays. The NenuFAR profile is also overlapping, but was visually aligned, as is the case for the non-simultaneous observations. The simultaneous observations are denoted with an asterisk (*).}
%     \label{fig: profiles}
% \end{figure}

% The profiles of the simultaneous observations in Fig.~\ref{fig: profiles} were aligned by correcting for geometric delays associated with the difference in location of the telescopes (taking right ascension to be $02^\mathrm{h}50^\mathrm{m}17\fs78$  and the declination to be $+58\degr54'01\farcs3$ as measured by \citealt{TBC+2018}). In addition, the dispersive delay associated with the propagation of the signal through the interstellar medium (ISM) was accounted for by using a DM of $45.281\pm0.003\ \mathrm{cm}^{-3}\mathrm{pc}$ \citep{TBC+2018}\footnote{There is no evidence for a change in the DM from measurements derived from recent DE601 and UK608 data.}. The long period of PSR~J0250+5854 means the uncertainty on the DM translates to an uncertainty on the dispersion delay between the highest frequency (FAST; 1250 MHz) and lowest frequency (NenuFAR; 57 MHz) observations of 3.9~ms, or around one pulse longitude bin at the highest resolution shown in Fig.~\ref{fig: profiles} (for the FAST and UK608 data) so is of little concern. The NenuFAR data was obtained during commissioning phase and so could not be aligned in this way, hence the peak of the profile was aligned visually with the UK608 profile peak.

% Only the GBT profile has a clear double-peaked profile morphology. Although single-peaked, the LOFAR Core profile, with a flat profile peak, and the asymmetric FAST profile can be taken as evidence for a more complicated profile structure. Inspecting the profiles in Fig.~\ref{fig: profiles}, it is evident that the profile width at frequencies below that of the FAST observation are significantly narrower, opposite to the expected behaviour by RFM. 

% The NenuFAR profile, corresponding to the lowest frequency, is broader again and distinctly skewed. Given the steep rise followed by an exponentially-decreasing tail, this can be attributed to scattering of the emission in the ISM. This is a strongly frequency-dependent effect with a power-law relationship between the scattering timescale and frequency, with a power law index of around $-4$ \citep{SDOx1980, PulsarAstronomy,GKK+2017}. This suggests that the scattering timescale for the NenuFAR data is around 50 times greater than at the UK608 centre frequency, which explains why only the NenuFAR profile is significantly affected. The NenuFAR profile is consistent with an intrinsic profile width which is equal to that observed at $\sim$150~MHz, albeit broadened by scattering. This is demonstrated in Fig.~\ref{fig: nenufar scattering} where the NenuFAR profile is compared with a von Mises function with a width equal to that of the UK608 profile, and convolved with an exponential scattering tail with an e-fold timescale of 0.1~s. This is consistent with the observed relationship between the DM and scattering timescale \citep[e.g.][]{BCC+2004,IJWx2019}. Therefore, scattering can fully explain the observed frequency evolution of the profile between 60 and 150~MHz (although given that its low signal-to-noise ratio makes it nearly impossible to resolve the profile reliably across the frequency band, the possibility of intrinsic profile evolution is not fully excluded). On the other hand, the high signal-to-noise (S/N) LOFAR Core profile also shows a somewhat elongated tail, but this could not be wholly attributed to scattering. This is because no significant evolution of the profile was observed in the frequency-resolved data, with the tail being present across the band. It therefore appears that this feature is intrinsic to the profile. The S/N is too low in the UK608 profile for the tail to stand out. So we conclude that only the NenuFAR profile shows clear evidence for being scatter broadened.
% \begin{figure}
%     \includegraphics[width=\columnwidth]{Figures/J0250/nenufar_scattering}
%     \caption[The scattered NenuFAR profile of PSR~J0250+5854]{The NenuFAR profile (top) compared to a smoothed UK608 profile convolved with a scattering tail (dashed model curve, red in the online version). No significant signal remains in the residuals (bottom).}
%     \label{fig: nenufar scattering}
% \end{figure}

% To confirm and quantify the pulse broadening at higher frequencies, we measured the profile widths as shown in Fig.~\ref{fig: profiles} by fitting von Mises functions to each profile using \textsc{psrsalsa}. This smooth mathematical description of the profile allows the width to be measured without being strongly affected by (white) noise. Two components were used to model the the higher S/N profiles (LOFAR Core, GBT) but including more than one component for weaker profiles would result in over-fitting. An uncertainties on each measurement was calculated using bootstrapping where for each iteration a rotated version of the baseline was added to the profile. This ensures that both the statistical noise arising from the white noise as well as residual baseline variations are accounted for. The estimated full width at half maximum ($W_{50}$) of the profiles in Fig.~\ref{fig: profiles} are shown in Tab.~\ref{tab: W50}.%The estimated full width at half maximum ($W_{50}$) of the profiles in order of increasing frequency are: $2.8\pm0.4\degr$ (57~MHz), $1.21\pm0.03\degr$ (149~MHz), $0.9\pm0.1\degr$ (150~MHz), $0.9\pm0.1\degr$ (154~MHz), $1.08\pm0.07\degr$ (350~MHz) and $2.4\pm0.1\degr$ (1250~MHz).
% \begin{table}
%     \centering
%     \caption[Measured profile width evolution with frequency]{The profile width at half maximum ($W_{50}$) of PSR~J0250+5854 as a function of frequency, measured by fitting von Mises functions to each profile shown in Fig.~\ref{fig: profiles}. These measurements are taken from the profiles as observed, and so includes the effect of scatter broadening in the case of the  NenuFAR profile.}
%     \label{tab: W50}
%     \begin{tabular}{lcc}
%         \hline
%         Telescope & Centre freq. (MHz) & $W_{50}~(\degr)$ \\
%         \hline
%         FAST & 1250 & $2.4\pm0.1$ \\
%         GBT & 350 & $1.08\pm0.07$ \\
%         DE601 & 154 & $0.9\pm0.1$ \\
%         UK608 & 150 & $0.9\pm0.1$ \\
%         LOFAR Core & 149 & $1.21\pm0.03$ \\
%         NenuFAR & 57 & $2.8\pm0.4$ \\ 
%     \end{tabular}
% \end{table}
% To further investigate the frequency evolution of the profile width, the widths were also determined after dividing the FAST data into four frequency sub-bands. Figure~\ref{fig: width evolution} shows the profile width against frequency for the profiles shown in Fig.~\ref{fig: profiles} (black) and the FAST sub-bands (blue).
% \begin{figure}
%     \includegraphics[width=0.75\textwidth]{Figures/J0250/thorsett_relation.png}
%     \caption[Profile evolution of PSR~J0250+5854 with observing frequency]{Evolution of the profile width of PSR~J0250+5854 with observing frequency. Points in black correspond to the profiles shown in Fig.~\ref{fig: profiles} and blue points are the profile widths of the four FAST sub-bands. The red and white dashed line represents the model of frequency evolution in Eq.~\eqref{eq: thorsett} which was fitted to the FAST sub-bands, GBT, UK608, and LOFAR Core data (NenuFAR was excluded because it is affected by scattering). A distribution of fits was calculated using bootstrapping techniques, and is represented by the red colour gradient. The black dotted lines indicate the 68~per~cent confidence interval of this distribution. The horizontal error bars indicate the bandwidth of a given observation.}
%     \label{fig: width evolution}
% \end{figure}
   
% The evolution of profile width with frequency $\nu$ was modelled using the relation
% \begin{equation}
%     \label{eq: thorsett}
%     W_{50} = A\nu^B + C,
% \end{equation}
% where $A$, $B$, and $C$ are constants \citep{Txxx1991, CWxx2014}. We fit the function in Eq.~\eqref{eq: thorsett} to the profile widths measured from the LOFAR Core, UK608, GBT, and FAST sub-band data. The NenuFAR profile was omitted to avoid scattering in the ISM affecting the results, and the DE601 and UK608 data were omitted because of their low S/N compared to the LOFAR Core data at a similar frequency. The horizontal error bars represent the bandwidth of a given observation. During fitting of Eq.~\eqref{eq: thorsett} the frequency of each observation was allowed to vary uniformly within these limits to account for the frequency-dependence of the profile width within the observed band. The distribution of fitted trend lines is shown in the red gradient plot in Fig.~\ref{fig: width evolution}. The black dotted lines bound the 68~per~cent confidence interval of the distribution, as a function of frequency. The red dashed line represents the optimal fit to the data, and the power-law exponent of Eq.~\eqref{eq: thorsett} is $1.9 \pm 0.4$. These findings are discussed further in Sec.~\ref{sec: J0250 - discussion}.


% \subsection{Polarisation and geometry}
% \label{sec: J0250 - analysis - polarisation and geometry}

% Polarisation calibration was performed on the FAST data using \textsc{psrchive}, making use of a pulsed noise diode signal injected into the 19-beam receiver. The multibeam receiver maintains a fixed orientation with respect to the sky during the observation by rotating within the focus cabin, so no parallactic angle corrections are required. After calibration, the polarised pulse profile of PSR~J0139+5814 (not shown, see also Sec.~\ref{sec: J0250 - observations}) is in excellent agreement with the results of \citet{GLxx1998} (publically available on the European Pulsar Network (EPN) database\footnote{\url{http://www.epta.eu.org/epndb/}}). The LOFAR Core data were not polarisation-calibrated using the LOFAR station beam model, but rather tied-array addition which incorporates data from different tiles and stations using the station calibration tables to account for the delays between them \citep[more detail can be found in ][]{SBG+2019}. The signs of Stokes V and the position angle curve had to be flipped in order to agree with convention \citep[e.g.][]{EWxx2001}, a correction that was also applied to the FAST data. The Faraday rotation measure towards PSR~J0250+5854, $\mathrm{RM}=-54.65\pm0.02$~rad~m$^{-2}$, was measured by applying RM synthesis \citep{BBxx2005} to the LOFAR Core polarisation data. This is consistent with the RM measured using the FAST data, although this has a larger uncertainty due to the higher observing frequency. Therefore, we de-Faraday-rotated the linear polarisation data for both observations using the LOFAR value.

% \begin{figure}
%     \includegraphics[width=0.75\textwidth]{Figures/J0250/polarised_profiles}
%     \caption[Polarised profiles of PSR~J0250+5854]{The polarised profile of PSR~J0250+5854, observed with FAST (upper plot) and LOFAR Core (lower plot). Total intensity is shown as the solid line, and linear and circular polarisation as dashed and dotted respectively. The lower panel of each plot shows the variation of position angle of linear polarisation with pulse longitude. The RVM (red curve in the online version) was fitted to the LOFAR Core data, and the same curve (with appropriate horizontal and vertical offset applied; see text) is shown for the FAST data.}
%     \label{fig: polarised profiles}
% \end{figure}
   
% In Figure~\ref{fig: polarised profiles} the polarised profile of PSR~J0250+5854 is shown as observed with FAST (first panel), and the LOFAR Core data (third panel). In both, the solid line is total intensity. The pulse profile has a moderate degree of linear polarisation (dashed), which was de-biased according to \citet{WKxx1974}. There is negative circular polarisation (dotted line) in the LOFAR observation, and a hint of the same in the FAST data. 

% The position angle (PA) $(\psi)$ as a function of pulse longitude is shown in the second and fourth panels of Fig.~\ref{fig: polarised profiles}, which relates to the Stokes $Q,\ U$ parameters via $\psi = 0.5 \arctan(U/Q)$. Its functional shape can be explained by the Rotating Vector Model \citep[RVM;][]{RCxx1969}, a geometric model which links the observed changes in PA with pulse longitude $(\phi)$ to the orientation of the magnetic field lines with respect to the observer. This therefore depends on the inclination angle $\alpha$ of the magnetic axis, and the impact parameter of the observer's line of sight, $\beta$, expressed as
% \begin{equation}
%     \label{eq: RVM}
%     \Delta\psi = \arctan\bigg(  \frac{\sin(\Delta\phi) \sin\alpha }{\sin(\zeta)\cos\alpha - \cos(\zeta)\sin\alpha\cos(\Delta\phi) }  \bigg),
% \end{equation}
% where $\Delta\psi = \psi - \psi_0$, $\Delta\phi = \phi - \phi_0$, and $\zeta = \alpha + \beta$. It describes a monotonic S-shaped curve where $(\phi_0,\ \psi_0)$ is the location of the inflection point. The inflection point is where the gradient of the curve is steepest, with a gradient equal to $\sin\alpha/\sin\beta$ \citep{Kxxx1970}. The relatively steep gradient of the PA curve ($\sim$55~deg~deg$^{-1}$) implies that $\beta\ll\alpha$, as expected for a detection of a slowly rotating pulsar with a narrow emission beam pointing along the direction of the magnetic axis.

% To fit the RVM, a grid search was conducted over pairs of $\alpha$ and $\beta$ values \citep[for details, see][]{RWJx2015a} for the LOFAR Core observation. The best fit to the observed PA points is shown in Fig.~\ref{fig: polarised profiles} for both the LOFAR and FAST data after applying an offset in PA to account for the fact that no absolute PA calibration has been performed, and allowing for a shift of the inflection point in longitude. As will be discussed in Sec.~\ref{sec: J0250 - discussion}, the offset in the PA inflection point ($1.3\pm0.1\degr$) between the FAST and LOFAR Core data is suggestive of emission at different radio frequencies originating at different heights in the magnetosphere. However, the functional shapes of the LOFAR and FAST PA data are identical within the errors, as expected when a dipolar field line configuration determines the shape. We therefore will only consider the RVM fit to the higher S/N LOFAR Core data.

% The goodness-of-fit is parametrised by the reduced-$\chi^2$ and its variation is shown in Fig.~\ref{fig: banana}. The darker shading corresponds to lower reduced-$\chi^2$ values and so a better fit. The black contours indicate $1\sigma$, $2\sigma$ and $3\sigma$ confidence intervals. As can be expected for a pulsar with a very small duty-cycle, $\alpha$ and $\beta$ are highly correlated due to the narrow range of PA points to which the RVM is fitted. The fit confirms that $\beta$ must be small ($<1.8\degr$), however the magnetic inclination $\alpha$ is unconstrained from RVM fitting alone.
% \begin{figure}
%     \includegraphics[width=\columnwidth]{Figures/J0250/banana.png}
%     \caption[The goodness-of-fit of the RVM to the PA curve of PSR~J0250+5854]{The goodness-of-fit (reduced-$\chi^2$) of the RVM to the PA curve as a function of ($\alpha$, $\beta$) space obtained for the LOFAR Core data is shown in grey-scale. The black contours correspond to reduced-$\chi^2$s of two-, three-, and four-times the minimum value. The green transparent regions are the ``allowed'' viewing geometries, which are constraints arising from the estimated emission height and observed profile width (see the main text for details including the assumptions made).}
%     \label{fig: banana}
% \end{figure}

% The measured profile widths provide additional information about the opening angle of the radio beam, how the line of sight cuts it, and the emission height. We assume that all radiation of a given frequency is produced at some height $h_\mathrm{em}$ in the magnetosphere in a circular region surrounding the magnetic axis. The emission beam is delimited by tangents to the last open field lines, forming a conal beam. In the small angle limit \citep[e.g. $h_\mathrm{em} \ll R_\mathrm{LC}$,][]{Rxxx1990} the half opening angle of the emission cone is
% \begin{equation}
%     \label{eq: cone angle}
%     \rho = \sqrt{\frac{9\pi h_\mathrm{em}}{2cP}}.
% \end{equation}
% This implies that the radio beam ought to widen with increasing emission height, and longer period pulsars can be expected to have narrower beams.

% The width of the pulse profile depends on how the line of sight cuts through the emission beam, as determined by $\alpha$ and $\beta$. \citet{GGRx1984} showed that the rotational phase range for which the line of sight samples the open-field-line-region, $W$, can be expressed as
% \begin{equation}
%     \label{eq: allowed geometry}
%     \cos\rho = \cos\alpha\cos(\alpha+\beta)+\sin\alpha\sin(\alpha+\beta)\cos\bigg(\frac{W}{2}\bigg).
% \end{equation}
% This means that a measurement of $W$ can help to constrain the parameters $\alpha$ and $\beta$, as well as $h_\mathrm{em}$ via $\rho$ \citep[see for example][]{RWJx2015a}. Here it is important to note that the open-field-line region does not necessarily emit over its full extent, hence the measured profile width does not necessarily correspond to $W$ as defined in Eq.~\eqref{eq: allowed geometry}.
    
% Fig.~\ref{fig: banana} highlights the geometries which are compatible with the observed pulse widths of the FAST observation ($W_{10}=4.3\pm0.2\degr$; the width of the profile as defined at 10~per~cent of the peak flux density) shown as the green shaded region. Here it is assumed that the beam is fully illuminated and that the emission height lies within the range of 200 to 400~km \citep[e.g.][]{MRxx2002, JKxx2019}. Moreover, $W$ as defined in Eq.~\eqref{eq: allowed geometry} is assumed to be between the measured $W_{10}$ and twice the distance between the PA curve inflection point and the furthest edge of the FAST pulse profile, in order to account for potential underfilling of the radio beam (see Sec.~\ref{sec: J0250 - discussion - geometry} for the  motivation). The uncertainty on the measured $W_{10}$, the emission height, and the filling fraction results in a range of contours in ($\alpha$, $\beta$) space defined by Eqs.~\eqref{eq: cone angle}~and~\eqref{eq: allowed geometry}. These contours (green region in the online version of Fig.~\ref{fig: banana}) show that $\beta$ is likely $\leq1.1\degr$, and suggests that the pulsar is relatively aligned (a small $\alpha$). The emission geometry of PSR~J0250+5854 is further discussed in Sec.~\ref{sec: J0250 - discussion}, where we conclude that although the underfilling of the beam could be somewhat more extreme compared to our assumptions in Fig.~\ref{fig: banana}, the result would only be a very modest extension of the allowed geometries towards slightly more aligned magnetic inclination angles.


% \subsection{Flux density spectrum}
% \label{sec: J0250 - analysis - flux}

% With a factor of $\sim$5 increase in spectral coverage with respect to \citet{TBC+2018}, the radio spectrum of PSR~J0250+5854 could be further quantified. Flux calibration of the FAST data was possible by utilising the observation of the nearby BL Lacertae object J0303+472, which was used as a reference source (see Sec.~\ref{sec: J0250 - observations}). This source, also known by the identifier 4C~47.08 \citep{VVxx2006}, has a known flux density of 1.8~Jy at a wavelength of 20~cm (approximately 1500~MHz, suitably close to the centre frequency of the FAST data at 1250~MHz) as listed in the VLA Calibrator List\footnote{\url{https://science.nrao.edu/facilities/vla/observing/callist}}. The NASA/IPAC Extragalactic Database (NED)\footnote{\url{https://ned.ipac.caltech.edu/}} entry for this object contains a list of flux densities of this source at different frequencies from the literature. This reveals a significant scatter in flux density measurements of observations at similar frequencies. Therefore, we assign an uncertainty of 50~per~cent to the flux density which is consistent with other work on pulsar flux density measurements \citep[e.g.][]{Sxxx1973}. The flux density calibration was performed using the \textsc{pac} routine in \textsc{psrchive}, and we measured a flux density of $4\pm2$~$\upmu$Jy for PSR~J0250+5854 at 1250~MHz. The S/N of the profile is also consistent with what is predicted for this flux density by the radiometer equation \citep[e.g.][]{Handbook} with known (zenith angle dependent) values for the gain $G = 14$~K~Jy$^{-1}$ and system temperature $T_\mathrm{sys} = 25$~K of FAST\footnote{see also Appendix 2 of \url{http://english.nao.cas.cn/focus2015/201901/t20190130_205104.html}} \citep{LWQ+2018}. This flux density is below the upper limits at a similar frequency based on non-detections with the Lovell and Nan\c{c}ay telescopes \citep{TBC+2018}.

% Fig.~\ref{fig: spectrum} shows the flux density of PSR~J0250+5854 as a function of observing frequency, and includes the flux densities previously measured by \citet{TBC+2018}. These previous measurements include detections with the GBT, LOFAR HBAs, and a flux density measurement obtained from the LOFAR Two-meter Sky Survey \citep[LoTSS;][]{SRB+2017}. 
% \begin{figure}
%     \includegraphics[width=\columnwidth]{Figures/J0250/spectral_index_turnover}
%     \caption[Radio spectrum of PSR~J0250+5854]{The flux density spectrum of PSR~J0250+5854, including upper limits (inverted triangles) from previous non-detections \citep{TBC+2018}. As in Fig.~\ref{fig: width evolution}, the horizontal error bars indicate the bandwidth of a given observation. This plot includes the previous flux density measurements from Tan et al., with the new addition of the NenuFAR and FAST measurements. A power-law relationship with a low-frequency turnover was fitted to the data, and the best fit is indicated by the red and white dashed line. The distribution of acceptable fits is indicated, similar to Fig.~\ref{fig: width evolution}. The NenuFAR bandwidth extends down to 19~MHz, beyond what is shown in the figure. The fitted parameters from Eq.~\eqref{eq: spectrum} are the scaling factor $b = 0.1^{+0.3}_{-0.0}$, turnover parameter $m = 2.1^{+0.0}_{-1.2}$, critical frequency $\nu_c = 94\pm24$~MHz, and spectral index $k = -3.5^{+0.2}_{-1.4}$.}
%     \label{fig: spectrum}
% \end{figure}
% The detection of PSR~J0250+5854 at 57~MHz using NenuFAR marks the lowest frequency detection that is published. The flux density of the pulsar at this frequency was estimated using the radiometer equation, and was found to be $1.7\pm0.9$~mJy, where we have again assigned a 50~per~cent uncertainty. In calibrating these data the elevation of the source and number of antennae in the array were taken into account as they affect the gain, as does the bandpass of the array. The sky background temperature was estimated to be 9050~K at the position of PSR~J0250+5854 (which dominates over the receiver temperature of 776~K), found by extrapolating the sky temperature measured at 408~MHz \citep{HSSW1982} with a spectral index of $-2.55$ \citep{LMOP1987, RRxx1988} to the centre frequency of 56.54~MHz. Full details of the NenuFAR flux calibration procedure are to be published in the instrumentation paper (Zarka et al., in prep.). This measurement indicates that the spectrum of PSR~J0250+5854 rolls over at low frequencies, and this explains the upper limit at a similar frequency reported by \citet{TBC+2018} based on LOFAR Low Band Antenna array observations. We further discuss the spectral shape in Sec.~\ref{sec: J0250 - discussion - flux}.

% \section{Discussion}
% \label{sec: J0250 - discussion}

% \subsection{Flux density and spectral index}
% \label{sec: J0250 - discussion - flux}

% PSR~J0250+5854 is weak at a centre frequency of 57~MHz as observed with the NenuFAR telescope (compared to its flux density at 150~MHz), which is the result of a spectral turnover (see Sec.~\ref{sec: J0250 - analysis - flux}). This is not unusual in the pulsar population: for example, \citet{BKK+2016} noted that 25~per~cent of their low-frequency sample were fitted by a broken power law with a turnover typically around 100~MHz. Furthermore, \citet{JSK+2018} noted that 21~per~cent of their sample deviates from a simple power law, exhibiting mainly broken power laws or low-frequency turnovers. The physical reasons for this are uncertain, but their analysis suggests that the deviations are partially intrinsic to the pulsar emission or because of magnetospheric absorption processes, and partially due to the environment around the pulsar or the ISM. 

% To quantify the spectral turnover of PSR~J0250+5854, a power-law with a low-frequency turnover was fitted to the data using the same model used by \citet{JSK+2018}, which is of the form
% \begin{equation}
%     \label{eq: spectrum}
%     S_\nu = b \bigg(\frac{\nu}{\nu_0}\bigg)^k \exp\bigg( \frac{k}{m} \bigg(\frac{\nu}{\nu_c}\bigg)^{-m}\bigg),
% \end{equation}
% where $\nu_0 = 500$~MHz is a constant (and arbitrary) reference frequency. The fitted parameters are $b$, a constant scaling factor; $k$, the spectral index; $\nu_c$, the turnover frequency; and $m$ which determines the smoothness of the transition. The value of $m$ is expected to be positive, and $\leq2.1$.

% With only one flux density measurement below the turnover frequency, the parameters are somewhat ill-defined. The optimal fit\footnote{The resulting probability density function of $m$ is highly clustered at the maximum allowed value of $2.1$ which was implemented as a prior. Therefore, no meaningful uncertainty on $m$ could be assigned.} (Fig.~\ref{fig: spectrum}, red line) is for an exponent $m=2.1$. This corresponds to the sharpest turnover allowed within the free-free absorption model (see \citet{JSK+2018} and references therein). The fitted spectral index of $k = -3.5^{+0.2}_{-1.4}$ is steep compared to the mean found for the pulsar population (\citealt{BLVx2013} found a mean spectral index of $-$1.4 with unit standard deviation, whilst \citealt{JSK+2018} found a mean of $-$1.60 with a standard deviation of 0.54). But other examples of such steep spectral indices exist, including PSR~J1234$-$6423 which has a broken power-law spectrum with a spectral index of $-3.8\pm0.5$ below $\sim$1700~MHz \citep{JSK+2018}.

% \citet{TBC+2018} noted that there are occasional bright pulses at 350~MHz (GBT data) in the leading component of the profile of PSR~J0250+5854. This behaviour was not seen in the LOFAR observations at around 150~MHz. If equally bright single pulses exist in the FAST frequency band, then they should be comfortably detectable above the level of the thermal noise. However, the residual baseline variations in the single pulse data are such that these bright pulses cannot be confidently detected. It therefore remains to be seen how the erratic nature of the single pulses evolves above frequencies of 350~MHz.



% \subsection{Profile width evolution}
% \label{sec: J0250 - discussion - width} 

% As can be seen in Fig.~\ref{fig: width evolution}, the profile width of PSR~J0250+5854 increases with observing frequency, from around $1\degr$ at 150~MHz to $2\degr$ at 1250~MHz. Eq.~\eqref{eq: thorsett} was fitted to the measured profile widths as a function of frequency, resulting in a power-law index $B = 1.9\pm0.4$ (see Sec.~\ref{sec: J0250 - analysis - profile widths}). Although it is unusual for pulsars to have a positive index, meaning that their profiles broaden with increasing frequency, there are other examples. \citet{CWxx2014} identified 29 pulsars out of 150 with such a positive index based on profiles from the EPN database. Of those, none have a value of the power-law index $B$ which is significantly larger than that of PSR~J0250+5854 given the relatively large uncertainty on both our measured value of $B$ and those measured by \citet{CWxx2014}. The index for PSR~J0250+5854 is consistent with 24 out of the 29 pulsars with a reported positive index. This increase in profile width with frequency is contrary to the expectation from RFM, and possible reasons are discussed here.

% Since one expects both the plasma density and the plasma frequency to decrease with increasing altitude (e.g. \citet{HAxx2001} and references therein; also \citet{GGMx2002}), higher frequency radiation can be expected to be produced closer to the neutron star if it is the result of plasma instabilities. Alternatively, if the radio emission is produced by curvature radiation from relativistic bunches of particles travelling along the magnetic field lines, the characteristic frequency of the radiation produced will also be higher near the surface of the neutron star given the smaller radius of curvature of the field lines  (e.g. \citealt{GLMx2004,DRxx2015} and references therein). As a consequence, the opening angle of the radio beam can be expected to be narrower at higher frequencies produced lower in the magnetosphere.  However, exceptions can be expected if profile components appear or disappear at different observing frequencies, for example if the fraction of the open-field-line region which is active is frequency dependent. \citet{PHS+2016} studied 100 pulsars and measured their profile widths at frequencies ranging from tens of megahertz up to 1400~MHz. They found that for the majority of cases, pulsars showed conventional-RFM characteristics, and sometime little frequency evolution of the pulse width at all. In a few cases (e.g. PSRs~B1541+09, B1821+05, B1822$-$09, and B2224+65), the profile width was seen to increase with frequency. In these examples, profile broadening was indeed caused by the emergence of new profile components as frequency increased. 

% A clear example of profile evolution with frequency due to multiple components is seen in PSR~B1919+21 as studied between 48 and 1700~MHz by \citet{HSH+2012}. The profile of this pulsar consists of two components which appear to move apart as frequencies increases. As this happens, one component becomes narrower, whilst the other broadens. Similarly, PSR~B0809+74 has a profile which was also modelled with two Gaussian components \citep{HSH+2012}. As frequency increases from 15 to 7850~MHz, the trailing component drifts to earlier phases, passing the fiducial component to end up on the leading side of the profile. Both these examples are difficult to explain by RFM alone, and a suggested explanation for PSR~B0809+74 was that refraction or a changing emission height could affect one component more than the other.

% The profile evolution of PSR~J0250+5854 is seemingly asymmetric as a function of frequency (Fig.~\ref{fig: profiles}), with the trailing half growing more rapidly than the leading half as frequency increases. As discussed in Sec.\ref{sec: J0250 - discussion - geometry} based on the shift in the inflection point of the PA swing as function of frequency, this asymmetry could be amplified by the emission height at LOFAR frequencies being higher than at FAST frequencies (as expected for conventional RFM). This suggests that there is a strong frequency dependence in which field lines are active which is driving the abnormal frequency evolution of the pulse width.

% However, the difference in the the PA curve inflection point location for the FAST and LOFAR polarised profiles implies that relativistic aberration and retardation effects (discussed in more detail in Sec.~\ref{sec: J0250 - discussion - geometry}) are present, meaning the LOFAR profile actually appears at slightly earlier longitudes than it ought by approximately $0.6\degr$. Correcting for this shift, the midpoints of the FAST and LOFAR profiles are then approximately aligned by eye. The FAST profile appears to be wider on both sides compared to the LOFAR profile as if it is a single component that has become broader rather than a distinct second component appearing in the trailing half. A simple application of Eq.~\eqref{eq: cone angle} might suggest that the same field lines are active and the broadening of the beam is due to the 1250~MHz emission being produced higher in the magnetosphere than the 150~MHz emission, however this cannot be made compatible with conventional RFM. 

% Observations at a frequency around 800~MHz could help reveal the reasons for the abnormal frequency evolution of the profile. The profile morphology of PSR~J0250+5854 is indeed complex, with a profile shape skewed in both the LOFAR Core and FAST observations. The GBT profile shows a distinct double-peaked structure with distinct behaviours since \citet{TBC+2018} noted that the stronger first component was caused by occasional strong individual pulses. At other frequencies no well separated profile components are observed. However, the flattened peak in the LOFAR Core profile is suggestive of two blended components of similar intensity.  This flattening was not visible for the profiles published in \citet{TBC+2018}, which is because of the lower S/N. By inspecting all available data, no significant profile shape variability has been detected in LOFAR Core observations of PSR~J0250+5854.

% \citet{CWxx2014} conclude that in a number of cases (PSRs~B1818$-$04, B1851$-$14, B1900+01, B1915+13, B2053+36, and B2217+47) the widening of the profile at higher frequencies could not be ascribed to structures consistent with the core-cone model \citep[e.g.][]{Rxxx1983a,Rxxx1983b, RRxx1990, Rxxx1993}. This also appears to be the case in a small sub-group of pulsars studied by \citet{PHS+2016} (PSRs~B0355+54, B0450+55, B1831$-$04, and B1857$-$26). Like PSR~J0250+5854, these pulsars do not show well-separated profile peaks at the highest frequencies. The expectation from conventional RFM is based on emission being produced over a wide range of altitudes, and each height producing narrowband emission. On the other hand, if a narrow range of emission heights generates broadband emission the observed spectrum will be different. This led \citet{CWxx2014} to suggest that fan beams could accommodate anti-RFM-like behaviour. Broadband emission is incorporated into the fan beam model \citep{Mxxx1987, DRDx2010, DRxx2012, DRxx2013, WPZ+2014} where emission is produced along magnetic flux tubes that extend out from the pole in a fan-like structure. Support for this beam structure is found in observations of the precessing pulsars J1141$-$6545 and J1906+0746 \citep{MKS+2010, DKC+2013}. Following the suggestion of \citet{Mxxx1987} that each flux tube may have its own spectrum, \citet{CWC+2007} argued that the emission spectrum may not be homogeneous across a flux tube. In particular, they argue that pulsars which show pulse broadening with increasing frequency may have a flattening emission spectrum away from the magnetic axis, as supported by their simulations. Broadband emission in flux tubes with a location-varying spectral index follows naturally from particle-in-cell simulations of vacuum-gap pair-production \citep{Txxx2010} which predict that the momentum spectrum of the secondary plasma is not necessarily monotonic as a function of height within the magnetosphere, and so a given observed frequency cannot be assigned to a unique altitude.


% \subsection{Emission height and viewing geometry}
% \label{sec: J0250 - discussion - geometry}

% \todo{We don't use the A/R effect to measure the emission height because we can't.}


% Constraining the viewing geometry is particularly interesting for slowly rotating pulsars to highlight differences with magnetars (see Sec.~\ref{sec: J0250 - discussion - compare}). As discussed in Sec.~\ref{sec: J0250 - analysis - polarisation and geometry}, RVM fitting to the PA swing supports the idea that the radio beam of PSR~J0250+5854 is very narrow by showing that the line-of-sight impact parameter $\beta \leq 1.8\degr$. However, the magnetic inclination angle $\alpha$ remained indeterminate.

% As seen in Fig.~\ref{fig: polarised profiles}, the inflection point of the PA swing occurs very close to the centre of the FAST profile, and is within the observable pulse at LOFAR frequencies. This implies that the emission height must be considerably smaller than the light cylinder radius. If this were not the case, relativistic aberration and retardation (A/R) effects would shift the inflection point outside the span of the pulse profile. \citet{BCWx1991} showed that the expected delay in pulse longitude between the inflection point and the longitude in the profile corresponding to the fiducial plane (the plane containing the magnetic and rotation axes) is $\Delta\phi = 4 h_\mathrm{em} / R_\mathrm{LC}$. This suggests that  $h_\mathrm{em} / R_\mathrm{LC} \lesssim 1$~per~cent ($10^{4}$~km) at FAST frequencies.

% This conclusion affirms that radio pulsars produce emission at an absolute emission altitude that is relatively constant across the population, rather than being at a constant fraction of $R_\mathrm{LC}$. Indeed extending the empirical model of RFM for non-recycled pulsars by \citet{KGxx2003} to the extremely long period of PSR~J0250+5854 suggests emission heights of several thousand kilometres -- from $(1.2\pm0.3)\times 10^3$~km at FAST frequencies to $(3\pm1)\times10^3$~km at NenuFAR frequencies -- consistent with the upper limit from our polarisation observations. Given the emission height is not at a constant fraction of $R_\mathrm{LC}$ there should be a period dependence of the pulse width \citep[e.g.][]{Rxxx1993}. Based on this relationship \citet{KJxx2007} proposed that the maximum emission height of radio pulsars at 1.4~GHz is around 1000~km, refined to an absolute height range of 200 to 400~km irrespective of pulse period \citep{JKxx2019, JSKx2020}. Again this is consistent with our results.

% There is a significant difference in the longitude of the inflection point of the PA curve between the LOFAR and FAST frequencies. This difference of $1.3\pm0.1\degr$ would suggest that the emission height at LOFAR frequencies is about 6000~km higher than at FAST frequencies. If the shift of the PA curve is interpreted by an emission height difference, it would affect the alignment of the profiles as well. The inflection point and the profile are shifted by equal amounts in opposite directions. Correcting for this shift in the profile would delay the LOFAR profile, making the widening of the profile at FAST frequency more symmetric. Here it should be noted that the inflection point lag is could also depend on the DM assumed -- a change in the DM of $\sim$0.4~cm$^{-3}$~pc would produce the same results, although the DM is known to a much higher precision than this and so we rule this out. The conclusion that the emission height could be significantly more narrow at FAST frequencies reinforces the conclusion that the filling factor of the beam is frequency-dependent. This filling factor impacts the ``allowed'' viewing geometries highlighted in the shaded region of Fig.~\ref{fig: banana}.

% In Sect.~\ref{sec: J0250 - analysis - polarisation and geometry} we assumed that the FAST profile occupies up to twice the distance between the PA curve inflection point and the furthest edge of the profile (the trailing edge). This assumes that because the emission height at FAST frequencies is argued to be low enough to make the A/R effects small, the PA curve inflection point coincides with the fiducial plane position. The span of $W$ (eq.~\ref{eq: allowed geometry}) used to make Fig.~\ref{fig: banana} extends from  a minimum of $4.1\degr$ (the conservative assumption that the beam is fully-filled, so free of making any assumptions about A/R effects, and narrower than the measured $W_{10}$ given its uncertainty) up to $6.9\degr$ (assuming that at least one edge of the profile corresponds to the last open field line region, and including the uncertainties in the profile width and position of the inflection point). Further to these considerations, it cannot be ruled out that the trailing edge of the profile does not reach the edge of the open field line region. This would correspond to a lower filling factor. However, even if the filling factor is decreased by a further factor of two, the minimum allowed $\alpha$ goes only from $\sim$20$\degr$ to $\sim$10$\degr$. So even a large uncertainty in the actual filling fraction has a marginal effect on Fig.~\ref{fig: banana} and the range of allowed $\alpha$.

% \subsection{Comparison to other slow pulsars and magnetars}
% \label{sec: J0250 - discussion - compare}

% The extremely long period of PSR~J0250+5854 places it on the far right-hand side of the $P$-$\dot{P}$-diagram, in an area largely inhabited by magnetars and XDINSs. Since it is believed that the pulse period is a key factor controlling the width of radio pulse profiles, it is worth comparing PSR~J0250+5854 with the radio-emitting magnetars and other slowly-spinning rotation-powered radio pulsars. This comparison can highlight what other parameters play a role in the radio beam geometry of these slowly rotating objects. Although the focus here will be the differences in profile widths, we note there are other differences such as the spectra of magnetars being radically different, and their radio emission being much more transient with periods of activity and strongly-changing profile shapes, as studied by \citet{SSW+2009, DJW+2018, LLD+2019, DLB+2019}. Despite the fact that magnetars form a distinct class of objects with much greater spin-down rates, and hence much higher rates of loss of rotational energy $\dot{E}$, they may evolve with time towards the parameter space occupied by the slow pulsars \citep[e.g.][]{VRP+2013}. However, the non-detection of PSR~J0250+5854 in X-rays makes such a connection uncertain \citep{TBC+2018}. Furthermore, PSR~J0250+5854 is located squarely in the Galactic plane (at Galactic coordinates $l=137.7\degr$, $b=-0.5\degr$) which argues that it is still relatively young.

% Aside from PSR~J0250+5854, the two other slowest-spinning known radio pulsars are PSRs~J2251$-$3711 ($P=12.1$~s), and J2144$-$3933 ($P=8.5$~s). There are five known magnetars for which pulsed radio emission has been detected: 1E~1547.0$-$5408 \citep{CRHR2007a}, PSR~J1622$-$4950 \citep{LBB+2010}, PSR~J1745$-$2900 \citep{EFK+2013}, XTE~J1810$-$197 \citep{CRH+2006}, and Swift~J1818$-$1607 \citep{ERB+2020, LSJB2020}. A summary of their properties is shown in Table~\ref{tab: comparison}.

% \begin{table*}
% 	\centering
% 	\caption[The three slowest rotation-powered pulsars and the five radio-detected magnetars]{Parameters (period $P$, spin-down rate $\dot{P}$) of the three slowest-spinning radio pulsars (top three sources) and the five magnetars (lower five sources) known to produce pulsed radio emission. The profile width of the radio pulsars are measured values of $W_{50}$ taken from this work and the referenced literature. The widths of the magnetar profiles are estimated based on the full extent over which significant emission was visible in the published profiles (at the reference frequency) in order to capture the complexity of the magnetar profiles. \newline \textbf{References:} (1) \citet{YMJx1999}; (2) \citet{MBM+2020}; (3) \citet{MKE+2020}; (4) \citet{CRHR2007a}; (5) \citet{CRJ+2008}; (6) \citet{LBB+2010}; (7) \citet{LBB+2012}; (8) \citet{EFK+2013}; (9) \citet{CRH+2006}; (10) \citet{CRJ+2007b}; (11) \citet{KSJ+2007}; (12) \citet{LLD+2019}; (13) \citet{ERB+2020}; (14) \citet{LSJB2020}; (15) \citet{CCC+2020}.}
% 	\label{tab: comparison}
% 	\begin{tabular}{lrrrrr} % 6 columns, alignment for each
% 		\hline
% 	    Object & $P$ (s) & $\dot{P}$ (ss$^{-1}$) & Profile Width ($\degr$) & Ref. Freq. (MHz) & References\\
% 		\hline
% 		PSR~J0250+5854          & 23.5 & $2.72\times10^{-14}$ &  $2.0$ & 1250 & This work\\
% 		PSR~J2144$-$3933        & 8.5  & $4.96\times10^{-16}$ &  $0.8$ & 1400 & 1, 2\\
% 		PSR~J2251$-$3711        & 12.1 & $1.31\times10^{-14}$ & $1.2$  & 1382 & 3\\
% 		\hline
% 		1E~1547.0$-$5408        & 2.1  & $2.32\times10^{-11}$ & $90$   & 6600 & 4, 5\\
%         PSR~J1622$-$4950        & 4.3  & $1.70\times10^{-11}$ & $190$  & 1400 & 6, 7\\
%         PSR~J1745$-$2900        & 3.8  & $6.80\times10^{-12}$ & $15$   & 2400 & 8\\
%         XTE~J1810$-$197          & 5.5  & $1.02\times10^{-11}$ & $35$   & 1400 & 9, 10, 11, 12\\
%         Swift~J1818.0$-$1607    & 1.4  & $9\times10^{-11}$    & $20$   & 1548 & 13, 14, 15\\
% 		\hline
% 	\end{tabular}
% \end{table*}

% A difference between magnetars and slow pulsars is that magnetars can have exceptionally shallow radio spectra. PSR~J1622$-$4950 has a flat spectrum between 1.4 and 24~GHz \citep{KJLB2011} similar to the spectra of 1547.0$-$5408 and XTE~J1810$-$197 \citep{CRJ+2007b, CRHR2007a, KSJ+2007}. In contrast, the spectral index of the slow pulsar PSR~J0250+5854 is exceptionally steep ($-3.5^{+0.2}_{-1.4}$ compared to the a mean of $-1.6$ for the non-recycled pulsar population). However, the spectral index of the magnetar Swift J1818.0$-$1607 is relatively steep as well at $-2.26^{+0.02}_{-0.03}$ \citep{LSJB2020}, which led them to believe there may be a link with the rotationally-powered PSR~J1119$-$6127 \citep[e.g][]{MPD+2017,DJW+2018}. Recently, \citet{CCC+2020} have shown that the spectral index of this object varies with time. 

% The three slowest-spinning radio pulsars have long periods and narrow, fairly simple pulse profiles. Measurements of $W_{50}$ are published for these profiles and are representative of the overall profile width. Despite being the slowest-spinning of the three, PSR~J0250+5854 has the widest profile by around a factor of two. If only the period determines the width, the inverse would be expected. This suggests that besides PSR~J0250+5854 (see Sec.~\ref{sec: J0250 - discussion - geometry}) underfilling of the beam also plays a role in the other slow pulsars. In contrast, the magnetars have fairly complex profiles with multiple components of differing intensity, which means that $W_{50}$ is not always a representative number for the overall width of the profile. Therefore, for the magnetars the profile width as quoted in Table~\ref{tab: comparison} spans the region where the emission in the published profiles is clearly distinguishable from the noise (rounded to the nearest five degrees).

% Looking at Table~\ref{tab: comparison} it is clear there is a stark contrast between the profile widths of the magnetars compared to the slow pulsars, much more than can be expected from just the differences in $P$ (and hence $R_\mathrm{LC}$). There are four potential geometric explanations for why magnetars have much wider profiles compared to the slow pulsars: 1) the three slowly rotating pulsars are observed with an extremely grazing line of sight with respect to the radio beam; 2) for the slowly rotating pulsars only a tiny fraction of the open-field-line region is active; 3) all magnetars have a magnetic axis almost aligned with the rotation axis; 4) magnetars produce radio emission much higher up in the magnetosphere. We will argue that only options 3) and 4) are viable, and that option 4) plays a more significant role. Option 1) relies on a very unlikely coincidence, and will not be considered further. 

% It was argued in Sec.~\ref{sec: J0250 - discussion - width} that only part of the open-field-line region is active for PSR~J0250+5854. However, this fraction needs to be very small for this and the other slow pulsars if it is to be the main reason why the slow pulsars have such narrow beams compared to the magnetars. This seems unlikely to be the case, as it would not explain why there are no slow pulsars with multiple narrow profile components spread over a similar fraction of the rotation period for which magnetars show emission. Instead, the structure of the profiles of the slow pulsars blend together in a single narrow profile, whereas the magnetar profiles have much more complex, multi-component structure. The magnetars often exhibit individual components which are much wider than the full profiles of the slow pulsars.

% If magnetars have very aligned radio beams with respect to their rotation axis (small $\alpha$), the observer's line of sight would spend a larger fraction of the time within the beam -- this can be seen by rearranging Eq.~\eqref{eq: allowed geometry} for a fixed cone angle $\rho$ and impact parameter $\beta$. If true, this suggests that these young objects have beams which become aligned much faster compared to what is believed to happen for normal pulsars \citep{TMxx1998,WJxx2008}. Polarisation studies of some of the radio magnetars suggest they may have aligned magnetic axes. This was suggested by \citet{CCR+2007} for XTE~J1810$-$197, although with orthogonal polarisation mode jumps it could be closer to orthogonality as well. \citet{KSJ+2007} preferred to interpret the polarisation of this magnetar with a non-aligned, offset dipole or a non-dipolar magnetic field configuration. An aligned magnetic field was suggested for 1E~1547.0$-$5408 \citep{CRJ+2008}, but RVM fitting for PSR~J1622$-$4950 by \citet{LBB+2012} suggested a configuration that is not particularly aligned. The slow pulsar PSR~J2144$-$3933 is thought to be a nearly orthogonal rotator \citep{MBM+2020}. An issue with attributing small $\alpha$ to magnetars is that it makes it difficult to reconcile with the large modulation of the thermal X-rays, as was highlighted for XTE~J1810$-$197 \citep{GHxx2007,PGxx2008} and 1E~1547.0$-$5408 \citep{IER+2010}.

% Eqs.~\eqref{eq: cone angle} and \eqref{eq: allowed geometry} can be used to estimate how extreme the alignment of the magnetars should be to explain their wide pulse profiles. Taking the mean magnetar period and median magnetar profile width from Tab.~\ref{tab: comparison} with an emission height of 400~km, such an object requires $\alpha\lesssim 11\degr$ to produce profiles of the observed width. Overall, the evidence from RVM fitting for magnetars having particularly small $\alpha$ values is mixed and X-ray data seem to preclude very small $\alpha$. Furthermore, if magnetars evolve into slowly-spinning rotation-powered pulsars these objects should have small $\alpha$ values and correspondingly wide profiles as well. For PSR~J0250+5854, for example, we have shown that it is unlikely that $\alpha$ is small. Therefore, it is concluded that the emission height must play a significant role in explaining the magnetar radio profile widths.

% If the emission heights are the dominant reason for the magnetars having wider radio profiles, they need to be $\sim$20 times larger (around 10,000~km) compared to the slowly-spinning rotation-powered pulsars. These large emission heights for the magnetars imply wide radio beams (Eq.~\ref{eq: cone angle}). However, alternatively, if their radio beams are confined by last open field lines which close within the light cylinder (as suggested by detailed simulations by \citealt{Sxxx2006}; see also the discussion of `Y-points' by \citealt{Cxxx2014}), or if magnetic field sweepback plays a large role \citep[e.g.][]{CRxx2012} then their polar beams will be wider as well. Therefore, one cannot distinguish between large emission heights and extended open field line regions \citep[e.g.][]{RWJx2015b, RWJx2015a}. In such a scenario, the open field line regions of magnetars would need to be $\sim$5 times larger than predicted for a static dipole field. In either case, it would imply that slowly-spinning rotation-powered pulsars have dramatically reduced beaming fractions compared to magnetars -- this could explain the deficit of observed slow pulsars near the death valley.

% The large implied difference in the beaming fraction between the two classes of objects is unexpected given the weak $\dot{P}$ dependence of the pulse widths observed for the normal pulsar population \citep[e.g.][]{KGxx2003, JKxx2019}. This implies that for these slowly rotating objects the role of $\dot{P}$ in governing the beaming fraction is much larger than for the normal pulsar population. This could potentially be facilitated by the incredible strengths of the magnetar magnetic fields.


% \section{Conclusions}
% \label{sec: J0250 - conclusions}

% We have obtained the highest- and lowest-frequency radio detections of PSR~J0250+5854, the most slowly rotating radio-emitting pulsar known, using simultaneous observations from 57~MHz to 1250~MHz. The highest frequency detection with FAST (1250~MHz) shows that the spectrum is exceptionally steep with a spectral index of $-3.5^{+0.2}_{-1.4}$ and the lowest frequency detection with NenuFAR (57~MHz) reveals a spectral turn-over below 95~MHz. The pulse profile shows a broadening at higher frequencies contrary to the expectations of radius-to-frequency mapping, which implies that the beam is underfilled at lower frequencies. The polarisation information of LOFAR Core data at 150~MHz and FAST data at 1250~MHz was used to constrain the viewing geometry. This shows that the line-of-sight impact parameter $\beta$ is very small, passing within $1.8\degr$ of the magnetic axis, and confirms that the radio beam is very narrow as expected for such a slow pulsar. Furthermore, the lack of a delay between the profile peak and position angle curve inflection point implies that the emission height of PSR~J0250+5854 at 1250~MHz is low, consistent with the 200 to 400~km range found for other non-recycled pulsars. Finally, we draw comparisons between other slow pulsars, PSR~J0250+5854, and the five known magnetars with pulsed radio emission which have the most similar pulse periods in the known pulsar population. We note that the profile widths of the magnetars are significantly broader than the normal slow pulsars -- we argue that whilst magnetic alignment in magnetars may play a role in explaining this, the main reason is likely to be either considerably expanded open field line regions or substantially larger emission heights for magnetars.

% %%%%%%%%%%%%%%%%%%%%%%%%%%%%%%%%%%%%%%%%%%%%%%%
% \section*{Acknowledgements}

% This research has made use of the NASA/IPAC Extragalactic Database (NED) which is operated by the Jet Propulsion Laboratory, California Institute of Technology, under contract with the National Aeronautics and Space Administration.  Pulsar research at Jodrell Bank Centre for Astrophysics and Jodrell Bank Observatory is supported by a consolidated grant from the UK Science and Technology Facilities Council (STFC). This research is also supported by the National Natural Science Foundation of China NSFC (Grant No. 11988101, No. U1938117, No. U1731238 and No. 11703003).

% This paper is based on data from the German LOng-Wavelength (GLOW) array, which is part of the International LOFAR Telescope (ILT) which is designed and built by ASTRON \citep{HWG+2013}. Specifically, we used the Effelsberg (DE601) station funded by the Max-Planck-Gesellschaft. The observations of the German LOFAR stations were carried out in the stand-alone GLOW mode which is technically operated and supported by the Max-Planck-Institut f\"ur Radioastronomie, the Forschungszentrum J\"ulich, Bielefeld University, by BMBF Verbundforschung project D-LOFAR III (grant number 05A14PBA) and by the states of Nordrhein-Westfalen and Hamburg.

% This paper is based on data obtained using the NenuFAR radio-telescope. The development of NenuFAR has been supported by personnel and funding from: Station de Radioastronomie de Nan\c{c}ay, CNRS-INSU, Observatoire de Paris-PSL, Universit\'e d'Orl\'eans, Observatoire des Sciences de l'Univers en r\'egion Centre, R\'egion Centre-Val de Loire, DIM-ACAV and DIM-ACAV+ of R\'egion Ile de France, Agence Nationale de la Recherche. We acknowledge the use of the Nan\c{c}ay Data Center computing facility (CDN - Centre de Donn\'ees de Nan\c{c}ay). The CDN is hosted by the Station de Radioastronomie de Nan\c{c}ay in partnership with Observatoire de Paris, Universit\'e d'Orl\'eans, OSUC and the CNRS. The CDN is supported by the Region Centre Val de Loire, d\'epartement du Cher. The Nan\c{c}ay Radio Observatory is operated by the Paris Observatory, associated with the French Centre National de la Recherche Scientifique (CNRS).

% This paper is based on data obtained with the International LOFAR Telescope (ILT) under project code DDT8\_004. LOFAR \citep{HWG+2013} is the Low Frequency Array designed and constructed by ASTRON. It has observing, data processing, and data storage facilities in several countries, that are owned by various parties (each with their own funding sources), and that are collectively operated by the ILT foundation under a joint scientific policy. The ILT resources have benefitted from the following recent major funding sources: CNRS-INSU, Observatoire de Paris and Universit\'{e} d'Orl\'{e}ans, France; BMBF, MIWF-NRW, MPG, Germany; Science Foundation Ireland (SFI), Department of Business, Enterprise and Innovation (DBEI), Ireland; NWO, The Netherlands; The Science and Technology Facilities Council, UK.

% J.W.T.H. acknowledges funding from an NWO Vici grant (``AstroFlash'').

% \section*{Data Availability}
% The data underlying this article will be shared on reasonable request to the corresponding author.









