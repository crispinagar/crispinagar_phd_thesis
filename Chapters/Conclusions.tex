\chapter[Conclusions]{Conclusions}
\label{chapt: conclusions}

In this thesis I have presented investigations into four pulsars, including two recently discovered objects, giving particular focus to their single pulse modulation properties and polarisation. In this final chapter I present a summary of the work completed and overall concluding remarks, as well as detailing the remaining questions and possible avenues of further research.

The aim of Chapter~\ref{chapt: B0031} was to investigate the asymmetry of the single-pulse polarisation properties of PSR~B0031$-$07, using archival data recorded with Parkes \citep{IWJ+2020}. This pulsar exhibits drifting subpulses with three distinct periodicities; $P_3 \approx 13P_1$ (mode A), $P_3 \approx 7P_1$ (mode B), $P_3 \approx 4P_1$ (mode C). Drift modes A and B were present in this data. This pulsar exhibits intensity-modulated orthogonal polarisation mode (OPM) transitions, something that has also been observed for a number of other pulsars \citep{RRS+2002,RRxx2003, Exxx2004,Ixxx2019}. The goal was to present a framework for how the asymmetry of the two linear OPMs and the total intensity emission can be explained if the drifting subpulses are produced by a circulating carousel of sub-beams \citep[][]{RSxx1975}, which predicts an intrinsic symmetry.

I present a model in which the emission from OPMs is coupled as they propagate through the magnetosphere, allowing power to be transferred between them. This takes place high in the magnetosphere where refraction is no longer a significant effect due to the decrease in plasma density. This scenario is supported by \citet{Pxxx2001}.  The position angle of the dominant OPM is that which is observed at a given longitude in the averaged emission, and only a small change in their relative intensity is required to cause an observed OPM transition. The strength of the mixing (and attenuation) of the two OPMs is assumed to be pulse longitude-dependent, which then allows asymmetries to arise in the observations. These effects are parametrised by a four-element `mixing matrix' which describes how the `intrinsic' emission changes as it propagates through the magnetosphere. This intrinsic emission could still have been affected by refraction at lower altitudes; for example the `ordinary' (O-) mode of wave propagation in a plasma is refracted, whereas the `extraordinary' (X-) mode is not \citep{ABxx1986}. This leads to a picture of a carousel which shows beamlets with distinct OPMs, but is still intrinsically axisymmetric \citep[e.g.][]{PLxx2000, ESLx2003, RRL+2006}. This pattern becomes baked-in as the refraction becomes the less dominant effect.


The linearly polarised emission was separated into the two OPMs under the assumption of incoherent mode addition \citep{MSxx2000} to produce the driftbands of the two observed OPMs. By exploiting the predicted symmetry of the underlying carousel, I was able to quantify the observed asymmetry by fitting the mixing matrix, and reveal the `intrinsic' OPM driftbands and carousel structures. The geometry of the carousel in PSR~B0031$-$07 was not known \textit{a priori}, so a `canonical model' was used that was motivated by literature values \citep{SMS+2007,MBW+2019}. In addition, an `atlas' of geometry parameters was explored to test the robustness of the model. Although degeneracies remain in the procedure and the results, matrices with a consistent structure were found across the canonical model and the atlas, showing that the qualitative features of the magnetospheric mixing are independent of the assumed carousel geometry and is different for the two drift modes. Furthermore, common features were found in the shapes of the intrinsic OPM driftbands and carousel sub-beams. For example, the sub-beams belonging to one of the OPMs in drift mode A are consistently more swept-out than the other OPM -- this suggests that this emission is more susceptible to distortion by refraction before mixing, potentially associating it with the O-mode \citep{ABxx1986}. 

The asymmetries could successfully be reproduced in emission from a symmetric carousel. This shows that applying this method leads to the expected qualitative results, although because of simplifications made in the model it is only expected to work quantitatively to first order. A key stage in the process is the method in which the OPMs are separated. There is evidence from the circular polarisation that coherent mode addition is occurring at a relatively low level, which is not accounted for. Incorporating this into the modelling requires many more degrees of freedom \citep[e.g.][]{Dxxx2017}, which would lead to additional degeneracies.

The conclusions of Chapter~\ref{chapt: B0031} are that magnetospheric mixing can explain the observed asymmetries in polarisation and total intensity in PSR~B0031$-$07, meaning that the carousel model is still a viable explanation for its drifting subpulses, which at first glance appear inconsistent with it. The mixing matrix is shown to be different for the two drift modes and appears independent of the geometry assumed for the carousel. This validates the assumption that mixing takes place at high altitudes, whereas the carousel structure is produced by `sparks' occurring low down in the polar cap. Further work could focus on trying to predict the magnetospheric mixing by linking it to plasma properties -- studies of the broader population of pulsars which exhibit drifting subpulses could then be performed to search for a correlation between their observed single pulse behaviour and theoretical magnetospheric properties.







Chapter~\ref{chapt: J1926} presents a study of PSR~J1926$-$0652, a pulsar with period $P_1 = 1.6$~s discovered with the new Five-hundred-metre Aperture Spherical radio Telescope (FAST) in China during its commissioning in 2017. Its single pulse data was recorded using the ultra-wide bandwidth (UWB) receiver on FAST between 270 and 800~MHz, and long-term monitoring and polarisation data was provided by Parkes. PSR~J1926$-$0652 has a wide, twin-peaked profile with two strong outer components and two weaker components nested within, connected by a bridge of emission. The single pulse data shows that it exhibits a number of interesting behaviours including nulling and drifting subpulses. It has a position angle curve that closely follows the rotating vector model \citep{RCxx1969, Kxxx1970}, and this in combination with its profile width of approximately $60\degr$ shows that its magnetic and rotation axes are somewhat aligned.

PSR~J1926$-$0652 produces bursts of emission of variable length, with a nulling fraction of around 40~per~cent \citep{ZLH+2019}. It shows drifting subpulses in both leading and trailing components with an average periodicity of $P_3 = 17.4\pm0.1P_1$. The drifting is stable in the longer bursts, but much more erratic in the shorter ones with a large variability in the shape and spacing of the driftbands. Fourier analysis was used to produce the subpulse phase tracks, revealing a large `kink' between the two components that make up the trailing half of the profile, and a similar, yet weaker, kink at lower frequencies between the profile components in the leading half. The driftband structure is consistent with emission from two nested, phase-locked carousels, with sub-beams that are azimuthally offset. The observed kink could also be explained in the alternative framework of non-radial oscillations of the neutron star \citep[e.g.][]{CRxx2004}. As shown in Chapter~\ref{chapt: B0031} the asymmetry of the driftbands is no longer considered to be inherently problematic for the carousel model. However, the lack of single pulse polarisation information makes it difficult to determine whether the mixing model could be applicable to PSR~J1926$-$0652.

A distinct difference in the single pulse emission properties was observed in the last active pulse before a null, where the leading profile component is significantly weaker than the trailing component. Although only seven transitions into a null were observed by FAST, Monte Carlo methods show that the extreme ratio between the power of the leading and trailing components has a 1 in 500,000 chance of occurring at random, based on the normal on-state behaviour. A similar feature was searched for in the first active pulses after a null, but was not found.  

Two possibilities for the difference of the last active pulse exist: either nulling is occurring at the end of a driftband \citep[as was observed for PSR~J1840$-$0840 by][]{GYY+2017}, which would point towards periodic nulling due to extinguishing sub-beams in the underlying carousel \citep[e.g.][]{HRxx2007, HRxx2009}; or the leading profile component begins to fade slightly earlier than the trailing component. It is not clear which is the correct scenario in PSR~J1926$-$0652 due to the erratic drifting in the short bursts and small sample of nulls, although the null does appear to occur simultaneously in both profile components which points to the latter. A slow transition into a null \citep[and slow recovery, as seen in PSR~B1944+17 by][]{DCHR1986} indicates that larger, global magnetospheric changes are taking place \citep{LHK+2010,MYxx2014}, which are responsible for the mode switching phenomenon. Overall, there is strong evidence that the drifting subpulse mechanism and nulling are connected in this pulsar, lending weight to the hypothesis that nulls are a form of extreme mode change \citep[e.g.][]{LKR+2002,WMJx2007, Txxx2010}.











In Chapter~\ref{chapt: J1518} I utilised the power of FAST to study the single pulses of the recycled millisecond pulsar (MSP) PSR~J1518+4904 ($P_1 = 41$~ms). The observations were performed in mid-to-late 2018 using the newly fitted 19-beam receiver during the commissioning of FAST. As for PSR~J1926$-$0652 there is no single pulse polarisation information. The profile of PSR~J1518+4904 from this observation was compared to archival data from the European Pulsar Network database, and shown to be consistent with Stokes $I$. The sensitivity of FAST revealed the existence of two previously unknown profile components in PSR~J1518+4904 which greatly extend the longitude range over which radio emission is detected. The weaker of these new components could potentially be emission from the opposite magnetic pole (implying a high magnetic inclination angle), but the other components form a single profile which is extremely wide ($\gtrsim$150$\degr$) which suggests a low inclination angle (or intrinsically extremely wide emission beam). In the two-pole scenario, this implies that the observer's line of sight only grazes the interpulse, so $\alpha$ may not be too large after all. 

PSR~J1518+4904 is one of only three MSPs known to exhibit drifting subpulses \citep{ESxx2003, LBJ+2016}. The individual pulses were detected with a good signal-to-noise ratio, allowing their modulation properties to be studied in unprecedented detail. Fourier analysis reveals the existence of three distinct periodicities in the main profile components, associated with different (but neighbouring) pulse longitude ranges in the longitude-resolved fluctuation spectrum. The two stronger features have frequencies of $P_1/P_3 = 0.38$ cycles per pulse period (cpp) and 0.28~cpp, whilst the third, weaker component is at 0.07~cpp. The three features were shown to be present in all eight sub-bands across the 1 to 1.5~GHz frequency range, and in three separate observations in July, August, and November 2018. The features at 0.28~cpp and 0.38~cpp were shown to have opposite drift directions. Furthermore, the feature at 0.38~cpp appears to change smoothly across the profile, decreasing slightly in peak fluctuation frequency over a pulse longitude range of $15\degr$ while the distribution of frequencies grows broader.

I considered the idea that different periodicities and drift directions are observed due to rapid drift mode changes that were unresolved by the Fourier analysis, which used blocks of 256 pulses. Mode switching is normally associated with slower, non-recycled pulsars, however profile morphology and polarisation changes have been observed in several MSPs \citep{KXC+1999}. Two distinct profile shapes due to mode changes were recently robustly identified in PSR~B1957+20 \citep{MKMP2018}. A change in the relative intensity of the different profile components where different modulation features occur was searched for in PSR~J1518+4904, however no difference was identified.

As no evidence for mode changing was found, the distinct single pulse periodicities therefore appear to be occurring simultaneously in PSR~J1518+4904. This, and in particular the longitude-dependence of $P_3$, present serious challenges for existing models of pulsar emission and how they can explain the observed pulse-to-pulse variability. To investigate this further, a proposal has been submitted to observe PSR~J1518+4904 again with FAST, since it is now fully commissioned, together with a sample of ten other recycled pulsars in order to determine whether other MSPs exhibit similar properties. This time we will record polarisation information as well to see if the curious single pulse modulation is not limited to Stokes $I$.












Chapter~\ref{chapt: J0250} is a report on observations of the slowest known pulsar, PSR~J0250$+$5854 ($P_1 = 23.5$~s), using FAST, two LOFAR international stations in the UK and Germany, and NenuFAR in France. This work has been submitted for publication in Monthly Notices of the Royal Astronomical Society \citep{AWB+2021}. The observations were performed simultaneously in order to investigate whether the pulsar produces emission above 1~GHz where it was so far undetected by \citet{TBC+2018}. Although PSR~J0250+5854 is an extremely slow rotator and lies beyond the pulsar `death line' according to several radio emission models \citep{CRxx1993,ZHMx2000}, its continuing emission can still be explained by the partially screened gap model \citep{Sxxx2013,MBMA2020}.

The pulsar was successfully detected by FAST at 1250~MHz and NenuFAR at 57~MHz, which extend the frequency coverage of this pulsar by a factor of around five. They reveal that it has a steep power-law spectrum with a turnover at low frequencies, which is not uncommon in the non-recycled population \citep{Sxxx1973}. Polarimetry data recorded with FAST (now fully commissioned) and archival LOFAR Core data shows that the observer's line of sight passes close to the magnetic axis and that the angle between the magnetic and rotation axes is less than $50\degr$. The data also show only very minor relativistic aberration and retardation effects, implying a low emission height. I argue that this is consistent with the range of less than 1000~km predicted for the normal, non-recycled population \citep{KJxx2007, JKxx2019}.

Measurements of the profile width of PSR~J0250+5854 show that it broadens significantly with increasing frequency, contrary to the expectations of radius-to-frequency mapping in a dipolar magnetic field geometry \citet{RSxx1975}. The implication is that the emission beam is severely underfilled (i.e. does not reach the edge of the open-field-line region) at low frequencies. This behaviour is consistent with the emergence of `conal outriders' in the core-cone model \citep{Rxxx1983a, Rxxx1983b, Rxxx1993}, where the central core beam has a steeper spectrum than the surrounding conal emission. The latter is therefore more prominent at higher frequencies, causing the profile to appear wider. This is not the only explanation for the profile frequency evolution -- \citet{MBMA2020} argue that the polar cap of PSR~J0250+5854 could support several sparks, but not necessarily enough to form a distinct core-cone structure. The widening profile could also be explained if broadband emission is produced across a range of emission heights, such as in the fan beam model \citep[e.g.][]{WPZ+2014}.

PSR~J0250+5854 lies close to the part of the $P$-$\dot{P}$ diagram inhabited by the magnetars, which also have long periods. Five magnetars are known to produce pulsed radio emission, and their profiles are drastically wider and more complex than the three slowest `normal' pulsars, which include PSR~J2251$-$3711 \citep[$P_1 = 12.1$~s][]{MKE+2020} and PSR~J2144$-$3933 \citep[$P_1 = 8.5$~s][]{YMJx1999}. Comparing different scenarios, I conclude that the beams of the magnetars are likely intrinsically wider than for the rotation-powered slow pulsars, either due to an increased emission height or because the last closed field lines close within the canonical light cylinder radius \citep{Sxxx2006, Cxxx2014}. This could be facilitated by their strong, non-dipolar magnetic fields.  The consequence is that the slow rotation-powered pulsars have much lower beaming fractions, which may explain their scarcity even after accounting for practical limitations on detection.

















The main conclusions of this thesis are the following. First, I conclude that there is a strong argument in favour of large scale magnetospheric effects being responsible for the observed emission properties of radio pulsars at the single pulse level. I have shown that longitude-dependent mixing and attenuation of orthogonal polarisation modes occurring at higher altitudes in the magnetosphere can explain the observed asymmetries in polarisation and total intensity in the drifting subpulses of PSR~B0031$-$07. This finding ensures that the carousel model remains a viable picture of the structure in the (low altitude) polar cap. Secondly, the work on PSR~J1926$-$0652 strengthens the argument for a connection between the drifting subpulse and nulling mechanisms, and is again in favour of global magnetospheric changes being responsible for mode switching. This pulsar should be revisited to obtain single pulse polarisation data. Thirdly, there is evidence of single pulse modulation in MSPs which could undermine the existing models that can explain non-recycled pulsars. A proposal has been submitted to perform further observations, with the aim to confirm this curious behaviour in PSR~J1518+4904 and establish to what extent it occurs in other recycled pulsars. Finally, the sensitivity of large telescopes such as FAST has shown to be crucial to observing the emission from older, slow pulsars such as PSR~J0250+5854. It is expected that future studies will greatly benefit from the sensitivity of such large instruments in recording detailed single pulse polarisation information, which will lead to the discovery of a wealth of new phenomena.